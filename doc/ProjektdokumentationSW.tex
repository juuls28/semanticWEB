\documentclass[a4paper,10pt,parskip]{article}
\usepackage{graphicx}
\usepackage{polyglossia}
\usepackage{xcolor} 
\usepackage{listings}
\setdefaultlanguage[spelling=new,babelshorthands=true]{german}

\usepackage{fontspec}
%\setmainfont{Source Serif Pro}
%\setsansfont{Source Sans Pro}

% ------
% Clickable URLs (optional)
\usepackage{hyperref}

\usepackage[left=2cm,right=2cm,top=3.0cm,bottom=3.0cm]{geometry}




\title{Projektdokumentation im Modul Semantic Web -- Zusammenhang von Terroranschlägen mit Witschaft und Politik}
\author{Hochschule für Technik, Wirtschaft und Kultur Leipzig\\Julius Seiffert - julius.seiffert@stud.htwk-leipzig.de}
\date{31.08.2019}


\begin{document}

\maketitle

\hyphenation{In-forma-tik}

\paragraph{Recherchefragestellung: }
Wie hängen Terroranschläge in Deutschland mit dem deutschen Aktienindex und den den deutschen Wahlverhalten zusammen.


\section{Inhaltliche Interpretation der Fragestellung}

Im Rahmen dieser Forschungsarbeit ist eine konkrete Interpretation, bzw. Schlussforderung natürlich nicht möglich. Alle Zusammenhänge die zwischen den Datenquellen gefunden wurden können durch Zufall entstanden sein.\\
Natürlich kann es bei dieser Fragestellung auch direkte Zusammenhänge geben. Terroranschläge können in der Gesellschaft einer Nation Angst und Hass schüren, was sich wiederum auf das Wahlverhalten auswirken kann. Auch können Terroranschläge bestimmte Bereiche der Wirtschaft finanziell zu Gute kommen, allerdings auch genauso Schaden. Beispielsweise könnte der Bereich der Waffen- und Militärausrüstung davon profitieren, oder  bei vorhandener Angst der Einzelhandel leiden.



\section{Relevante Datenquellen}

Relevante Datenquellen waren die Terrordatenbank der University Maryland, die Website Yahoo Finance, welche historische Daten des DAX zum Download zur Verfügung stellt und die Sonntagsfrage des FORSA Instituts, welches nahezu jeden Sonntag eine Wahlumfrage erhebt.

\subsection{Terrordatenbank der University Maryland}

Die Datenbank der University Maryland beinhaltet globale Terroranschläge von 1970 bis 2017. Auf der Website werden die Daten als CSV-Datei zum Download angeboten und können auch direkt über eine Tabelle auf der Website angezeigt werden.

\vspace{0.5cm}
\begin{tabular}{l|p{9cm}}
	Link & \url{https://www.start.umd.edu/gtd/about/} \\
 	Datenformat & CSV, HTML \\
 	Schnittstelle & Webscarper, manueller Download \\
 	Lizenz & \url{https://www.start.umd.edu/gtd/terms-of-use/} \\
 	Open Data & $\star\star\star$ \\
\end{tabular}

\subsection{Yahoo Finance}

Die Website Yahoo Finance stellt sämtliche Börsenkurse und auch historische Daten davon bereit.\\
Yahoo Finance hat bis zum März 2019 eine API-Schnittstelle der verfügbaren Finanzdaten angeboten. Leider wurde diese Schnittstelle eingestellt und die Daten können nur durch einen Webscarper, oder einen manuellen Download heruntergeladen werden. Eine Automatisierung des Downloads gestaltet sich durch Sicherheitsmaßnahmen der Website als schwierig. Aufgrund dessen wurden die die Tageshöchstsätze des DAX manuell als CSV-Datei heruntergeladen, welche bis zum 01.01.2004 zurückgeht.

\vspace{0.5cm}
\begin{tabular}{l|p{9cm}}
	Link & \url{https://de.finance.yahoo.com/} \\
 	Datenformat & CSV, HTML \\
 	Schnittstelle & manueller Download \\
 	Lizenz & \url{https://policies.oath.com/ie/de/oath/terms/otos/index.html} \\
 	Open Data & $\star\star$ \\
\end{tabular}

\subsection{Sonntagsfrage FORSA}

Das FORSA Institut stellt zu jeden Sonntag die Sonntagsfrage. Bei dieser Frage werden die Befragten gefragt, welche Partei sie wählen würden, wenn am nächsten Sonntag die Bundeswahl wäre. Diese Daten gehen bis 1998 zurück. Es gibt allerdings auch Zeiträume, in denen keine Daten vorhanden sind.\\
Die Daten werden auf der Website \url{https://www.wahlrecht.de/} in HTML-Tabellen bereitgestellt.

\vspace{0.5cm}
\begin{tabular}{l|p{9cm}}
	Link & https://www.wahlrecht.de/umfragen/forsa.htm \\
 	Datenformat & HTML \\
 	Schnittstelle & Webscarper \\
 	Lizenz &  \\
 	Open Data & $\star\star$ \\
\end{tabular}

\section{Vokabular}
Für den Aufbau der Datenbasis wurde das RDF-Vokabular verwendet. Der Präfix zu dem RDF-Vokabular ist \textit{http://www.w3.org/1999/02/22-rdf-syntax-ns}\\

Für die spezifischen Daten wurde ein neues Vokabular erstellt. Dieses wurde unter dem Präfix \textit{https://github.com/juuls28/semanticWEB} verwendet.\\ 
\subsection{Terroranschlag}
Die Daten der Terroranschläge sind durch Ihre ID gekennzeichnet.\\
Aus der Terrordatenbank lassen sich viele Daten zu jedem Terroranschlag auslesen. Hier wurden nur die wichtigsten extrahiert.\\
\\
\textbf{happenedOn} beschreibt das Datum im Format yyyy-MM-dd an dem das Ereignis geschehen ist.\\
\\
\textbf{inCountry} beschreibt das Land in dem das Ereignis passiert ist als Zeichenkette.\\
\\
\textbf{inCity} beschreibt die Stadt in dem das Ereignis passiert ist als Zeichenkette.\\
\\
\textbf{hasFatalitites} beschreibt die Anzahl der getöteten Personen als ganze Zahl.\\
\\
\textbf{hasInjuries} beschreibt die Anzahl der verletzten Personen als ganze Zahl.\\
\\
\subsection{DAX-Kurs}
Die Daten des Aktienkurses sind durch das Wort DAX und das Datum eindeutig identifizierbar. Die Ressourcen haben daher die Form DAXyyyy-MM-dd.\\
Bei den historischen Daten des DAX-Kurses wird das Datum und der zugehörige Eröffnungskurs beachtet.\\
\\
\textbf{happenedOn} beschreibt das Datum im Format yyyy-MM-dd an dem das Ereignis geschehen ist.\\
\\
\textbf{value} beschreibt den Wer des Eröffnungskurses am Tag der Ressource als Fließkommazahl.\\
\\



\section{Extraktion relevanter Daten und Import in einen Triplestore }

Die Extraktion der Daten erfolgt über die verschiedenen SPARQL-Endpunkte.

\subsection{Extraktion Professorenkatalog}

\subsection{Extraktion GND}


\vspace{0.5cm}\textcolor{blue}{\textbf{Es handelt sich hier um ein Muster, das Dokument ist daher unvollständig...}}

\section{Verlinkung von Ressourcen}

Die Verlinkung erfolgt über die GND. dabei werden im Triple-Store \verb+owl:sameAs+ Triple\footnote{OWL Web Ontology Language Reference owl:sameAs: \url{http://www.w3.org/TR/owl-ref/\#sameAs-def}} hinzugefügt. Die Listing \ref{list:sameAs} zeit dies an einem Beispiel:

\begin{lstlisting}[caption={Beispiel für die Verwendung von owl:sameAS}, label={list:sameAs}]
http://catalogs-professorum.org/lipsiensis/Schuecking_144 owl:sameAs http://d-nb.info/gnd/117124931.
\end{lstlisting}


\vspace{0.5cm}\textcolor{blue}{\textbf{Es handelt sich hier um ein Muster, das Dokument ist daher unvollständig...}}

\section{Anfrage an die Forschungswissensbasis}

\subsection{SPARQL-Anfrage}

\subsection{Ergebnis der Anfrage}


\vspace{0.5cm}\textcolor{blue}{\textbf{Es handelt sich hier um ein Muster, das Dokument ist daher unvollständig...}}


\section{Interpretation und Zusammenfassung}

\vspace{0.5cm}\textcolor{blue}{\textbf{Es handelt sich hier um ein Muster, das Dokument ist daher unvollständig...}}

\bibliographystyle{alpha}
\bibliography{ProjektdokumentationSW}



\end{document}
